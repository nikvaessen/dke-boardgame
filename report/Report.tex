\documentclass[a4paper]{article}

%% Language and font encodings
\usepackage[english]{babel}
\usepackage[utf8x]{inputenc}
\usepackage[T1]{fontenc}

%% Sets page size and margins
\usepackage[a4paper,top=3cm,bottom=2cm,left=3cm,right=3cm,marginparwidth=1.75cm]{geometry}

%% Useful packages
\usepackage{amsmath}
\usepackage{graphicx}
\usepackage[colorinlistoftodos]{todonotes}
\usepackage[colorlinks=true, allcolors=blue]{hyperref}

%% Title
\title{AI implemenetations for Hex}
\author{Alexander Steckelberg, Nik Vaessen, Jose Luis Velasquez, Jeroen Vermazeren, Thomas Wall, Xavier Weber}

\begin{doucument}

    \maketitle
    \tableofcontents

    \begin{abstract}
        Here we write the abstract
    \end{abstract}

    \section{Introduction}
    The board game Hex comprises of a board of hexagons, typically 11x11 in size, with the aim of creating a bridge from either top to          bottom or left to right. The game was first created in 1942, and improved into the version we know today in 1947. As the game is        fully determined, as connecting one player's path will block the other, it is an ideal candidate to create an AI for.

    In this report, several methods of implementing an algorithm to play the game will be explored and explained. These methods will            also be tested by limit factors within them and by their performance against other algorithms and human players.
    \subsection{What is Hex?}
    \subsection{History}

    \section{Our implementation}
    \subsection{Hex}
    \subsection{A.I. Players}

    \section{Methods}
    %% here we can discuss what type of experiments we will be doing

    \section{Results}
    %% here we se the result of the experiments

    \bibliographystyle{alpha}
    %this references the bib file
    \bibliography{biblio}




\end{document}
