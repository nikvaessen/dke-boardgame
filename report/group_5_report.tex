\documentclass{ba-kecs}
\usepackage{graphicx}



\title{AI implementations for Hex}
\runningtitle{AI for Hex}
\author{A. Steckelberg, N. Vaessen, J.L. Velasquez, J. Vermazeren, T. Wall, X. Weber}

\begin{document}

\maketitle

\begin{abstract}
 Hex is a classic board game invented in 1942 by Piet Hein and independently by John Nash in 1948. In this paper there is research into alpha-beta  and Monte Carlo Tree Search Hex players. We compare different evaluation functions which includes Dijkstra, Electric Circuit besides this we compare Monte Carlo Tree Search specific heuristics.   
\end{abstract}

\section{Introduction}
Hex is a well-known board game first created in 1942 by Piet Hein, a Danish physicist, with later improvements being made by John F. Nash in 1948. It gained the name Hex in 1952 when a version of the game was release by the firm Parker Brothers, Inc \cite{gardener1959hex}. The basic idea of Hex is to create a bridge across a diamond shape board of hexagons. The generally accepted normal size of the board is 11 x 11, however it can be played on boards of differing sizes. One player will try to create a bridge from the top of the board to the bottom, while the other player tries to make one from left to right. The game is fully deterministic, as by finishing one player’s bridge will always block the other player. As the first player is always at an advantage, Hex has a unique rule to address this. Called the Pie Rule, it states that after the second player has had their first move, they may switch the placement of the first two tiles.\\
The history of creating algorithms to play Hex is rich with successful attempts, all using a different approach. In this report, several of these attempts will be implemented and tested against both each other, while also seeing if limiting factors in these algorithms (time per move, tree depth, etc.) have major effects on the results.



\section{Rules and Algorithms of Hex}
\subsection{Rules}
Rules for Hex
\subsection{Alpha-beta Hex Players}
Explanation of alpha-beta 
\subsection{Electric circuit}
Explanation of the electric circuit of Anshelivic
\subsection{Monte Carlo Tree Search}
Article used for MCTS \cite{arneson2010monte}
Article used to understand MCTS \cite{browne2012survey}
\subsubsection{Select}
\subsubsection{Expand}
\subsubsection{Simulate}
\subsubsection{Backpropagate}
Explanation of general MCTS
\subsection{Improvements on general MCTS}
\subsubsection{UCT}
Explanation of Upper Confindence Treshold. Multi armed bandit problem
\subsubsection{AMAF/RAVE}

Explanation of either AMAF or RAVE \cite{gelly2011monte}
\subsubsection{Parallelisation}
Explanation of different parallelisation methods

\section{Experiments}
Explanation of experiments and results, comparing different parameters
\section{Results}
Explanation of results of the experiments
\section{Conclusion}
Conclusion on the explanation of the result of the experiments
\bibliography{biblio}



\end{document}

